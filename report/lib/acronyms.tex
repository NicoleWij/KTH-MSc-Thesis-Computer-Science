%%% Local Variables:
%%% mode: latex
%%% TeX-master: t
%%% End:
% The following command is used with glossaries-extra
\setabbreviationstyle[acronym]{long-short}
% The form of the entries in this file is \newacronym{label}{acronym}{phrase}
%                                      or \newacronym[options]{label}{acronym}{phrase}
% see "User Manual for glossaries.sty" for the  details about the options, one example is shown below
% note the specification of the long form plural in the line below
\newacronym[longplural={Debugging Information Entities}]{DIE}{DIE}{Debugging Information Entity}
%
% The following example also uses options
\newacronym[shortplural={OSes}, firstplural={operating systems (OSes)}]{OS}{OS}{operating system}

% note the use of a non-breaking dash in long text for the following acronym
\newacronym{IQL}{IQL}{Independent Q‑Learning}

% example of putting in a trademark on first expansion
\newacronym[first={NVIDIA OpenSHMEM Library (NVSHMEM\texttrademark)}]{NVSHMEM}{NVSHMEM}{NVIDIA OpenSHMEM Library}

\newacronym{KTH}{KTH}{KTH Royal Institute of Technology}

\newacronym{LAN}{LAN}{Local Area Network}
\newacronym{VM}{VM}{virtual machine}
% note the use of a non-breaking dash in the following acronym
\newacronym{WiFi}{Wi‑Fi}{Wireless Fidelity}

\newacronym{WLAN}{WLAN}{Wireless Local Area Network}
\newacronym{UN}{UN}{United Nations}
\newacronym{SDG}{SDG}{Sustainable Development Goal}

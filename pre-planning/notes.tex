- use klimra for help to get stuff explained, let her know how it goes with the supervisor.
- I'm gonna argue that maybe the problem has been studied before, but there are some problems. there is some tension. everything fine, but there is a problem that we are going to adress. Describe a bit more about the research gap. is there something in my method I am inspired by other research papers I can describe those as well. Write a paragraph about the paper as you read them. I'm gonna describe my method blabla but I'm going to have to explain it in the thesis. Litreview and also in background. What do I need to know to appreciate what I did. Pre-requisites: anything I need to know that we didn't study in school. 
- describe previous results if there is something. dont explain all of them, just what is most relevant.
- of course follow the template, but I'm going to document my thinking process. that's the most important thing, focus. I'm gonna have new problems to work on, did I have to understand something new like a new algorithm? write about that. The focus is on the WORK I DID. there will be a result at the end. what did I try? what worked? what didn't work? Don't worry too much about what each component is supposed to mean. in general: you can have like a research diary/journal where I write down every day that I work on the project and just write down what I want to do and where I got stuck. useful since I might forget why I did x and not y, I can reference back to the reasoning behind it from the diary. this also helps when you meet your supervisor: this is what I tried, I thought about this, it's not really working out. Gonna feel like administrative work, write it down very quickly for later. But there are moments where it can make a difference. DIARY.
- a lot of atonomy, independent. Expand on the research gap, if there are related papers that have done similar things, explain them a little. research gaps and whatever techniques I think is important to include. 
- my job is to focus on the work, not the template.
- as I read the papers, I'm gonna go through papers and document them. good to have an overview of things. skeleton of thesis -> think of a 20 minute talk. 
- what are the general trends? people have worked on this. not explain so much papers, but argue why this is important to study. This is an important problem, people have worked on it, here's the gap.
- The thesis will take some time, depends on writing speed. Don't stop writing.
- Have september in mind, that's fine, do as much work as I can.
- meet again in week 27. I'll send the draft of the pre-study before I leave on vacation.
- it's a good idea to aim for the pre study before I leave. If I send it to her earlier we might be able to meet before I leave.
- Technical discussion after I come back. she can be a bit of a springboard, an academic, what would an academic ask about this. From a student perspective you get asked a lot of questions, scary, but it comes from people who are just interested.
- plan: Pre study -> whenever we meet, we talk a little about the work as well. 
- Let her know how the work environment is.
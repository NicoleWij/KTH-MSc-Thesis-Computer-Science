\documentclass[11pt]{article}
\usepackage[a4paper, margin=2.5cm]{geometry}
\usepackage{titlesec}
\usepackage{enumitem}
\usepackage{hyperref}

\title{Individual Plan}
\author{}
\date{}

\begin{document}

\maketitle

\noindent\textbf{Comment to the student:} the individual plan needs to be prepared within a few weeks after project start, and has to be accepted by the examiner. The document should be 3--4 pages long.

\vspace{1em}
\section*{PROJECT INFORMATION}

\begin{itemize}[leftmargin=*, label={}]
  \item Preliminary title, that indicates what the degree project will be about.
  \item The name and e-mail address of the student
  \item The name of the examiner at KTH
  \item The name of the supervisor at KTH
  \item The name and e-mail address of the supervisor, if the thesis is performed outside KTH
  \item Current date
  \item Keywords
\end{itemize}

\section*{BACKGROUND \& OBJECTIVE}

\begin{itemize}[leftmargin=*, label={}]
  \item Description of the area within which the degree project is being carried out with connection to scientific and/or societal interest.
  \item Description of the interest of the organization or company who provided the assignment.
  \item The high level objective of the project, the desired outcome from the perspective of the assignment provider.
  \item The background knowledge required to carry out the project.
\end{itemize}

\section*{RESEARCH QUESTION \& METHOD}

\begin{itemize}[leftmargin=*, label={}]
  \item \textbf{QUESTION:} State the question that will be examined. Formulate is as an explicit and evaluable question. State your hypothesis.
  \item \textbf{Objectives:} Break down the research questions to measurable objectives.
  \item \textbf{Tasks:} Describe the tasks that are necessary to reach the objectives. For each task, describe the challenges it involves.
  \item \textbf{Method:} Describe the method/s that will be followed. Explain why they are appropriate for the project or for the specific tasks.
  \item \textbf{Ethics and Sustainability:} Does the project address questions of ethics or sustainability? Does the project raise ethical or sustainability questions? If yes, how could these be handled?
  \item \textbf{Limitations:} Define the limitations on what is to be done (so that it is clear what is not included in the degree project).
  \item \textbf{Risks:} Explain what can go wrong and delay or make the project impossible to conclude. Explain how you will deal with these problems.
\end{itemize}

\section*{EVALUATION \& NEWS VALUE}

\begin{itemize}[leftmargin=*, label={}]
  \item \textbf{Evaluation:} How is it determined if the objectives of the degree project have been fulfilled and if the research question has been adequately answered? What kind of qualitative or quantitative measures can be defined and evaluated?
  \item \textbf{Expected scientific results:} How is the work scientifically relevant?
  \item \textbf{The work's innovation/news value:} Why does someone want to read the finished work? And who are these people?
\end{itemize}

\section*{PRE-STUDY}

\begin{itemize}[leftmargin=*, label={}]
  \item Description of the literature studies. What areas will the literature study focus on? How shall the necessary knowledge on background and state-of-the-art be obtained? What preliminarily important references have been identified?
\end{itemize}

\section*{CONDITIONS \& SCHEDULE}

\begin{itemize}[leftmargin=*, label={}]
  \item List of the resources are needed to solve the problem. This can be technical equipment, software, or data, but also experiment and interview subjects.
  \item Describe the way the external supervisor will be involved in the project.
  \item Provide a project timeline, specifying the main tasks and the time allocated for them, milestones (time of achievement of intermediate goals)
\end{itemize}

\section*{REFERENCES}

List references, including information Authors, Title, Journal/Book/Conference/Website, Publication date.

\end{document}

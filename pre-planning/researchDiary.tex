\documentclass[11pt]{article}
\usepackage[a4paper, margin=2.5cm]{geometry}
\usepackage{datetime2}
\usepackage{enumitem}
\usepackage{xcolor}
\usepackage{titlesec}
\usepackage{hyperref}
\hypersetup{colorlinks=true, linkcolor=blue, urlcolor=blue}

\titleformat{\section}{\normalfont\Large\bfseries}{\thesection}{1em}{}
\titleformat{\subsection}{\normalfont\large\bfseries}{\thesubsection}{1em}{}

\newcommand{\entrydate}[1]{\section*{#1}}
\newcommand{\subheading}[1]{\subsection*{\textcolor{gray}{#1}}}

\title{Research Diary – Nicole Wijkman}
\author{}
\date{}

\begin{document}

\maketitle

\tableofcontents
\newpage

% EXAMPLE ENTRY
\entrydate{2025-05-30}

\subheading{Goals for Today}
\begin{itemize}[leftmargin=*]
    \item Find eventual sources.
    \item Read through my already collected sources, write a paragraph about each.
    \item EXTRA: Read through 1.2 Background, comment on potential additions/fixes.
\end{itemize}

\subheading{What I Did}
\begin{itemize}[leftmargin=*]
    \item Read paper: Shah et al., 2022 on time-series DBs.
    \item 
    \item 
\end{itemize}

\subheading{What Worked}
\begin{itemize}[leftmargin=*]
    \item Paper was very clear and helped justify Timestream as a choice.
    \item Realized CloudWatch can provide native metrics for all three storage options.
\end{itemize}

\subheading{What Didn't Work}
\begin{itemize}[leftmargin=*]
    \item Still unclear how to simulate real-time API rate limits using k6.
    \item AWS Timestream documentation lacks examples for write throughput benchmarking.
\end{itemize}

\subheading{Next Steps}
\begin{itemize}[leftmargin=*]
    \item Look at Marcu et al., 2022 on push vs. pull.
    \item Try setting up k6 with synthetic endpoints for testing.
\end{itemize}

\subheading{Reflections}
Small win today: I'm starting to get a better sense of where the research gap is. Even though the problem has been looked at before, no one is isolating the architecture/storage combination under identical ingestion conditions. That’s my niche.

\newpage

% Add more entries like this:
% \entrydate{2025-06-11}
% ...

\end{document}
